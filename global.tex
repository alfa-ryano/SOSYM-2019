\NeedsTeXFormat{LaTeX2e}[1995/12/01]
\documentclass[draft]{ltxguide}[1995/11/28]
%\usepackage{draftcopy}
  \usepackage[logonly]{trace}

\newcommand{\SJour}{\textsc{SVJour}}

\title{The \SJour\ document class users guide\\supplement
for\\\textit{global use}}

\author{\copyright~1998, Springer Verlag Heidelberg\\
   All rights reserved.}

\date{29 October 2003}

\begin{document}

\maketitle

\section{Introduction}
\label{sec:intro}
This document describes the \textit{global} option for the \SJour\
\LaTeXe\ document class. It is meant for journals (of one- or two-column
style) that have not yet been supported via a particular class option.

For details on manuscript handling and the review process we refer to
the \emph{Instructions for authors} in the printed journal you are
writing your article for. For style matters please consult also previous
issues of ``your'' journal.

\section{Initializing the Class}
\label{sec:opt}

Please check the home page of your very journal via\\[3pt]
\null\hspace{5mm}\hbox{\vtop{\hbox{\verb|http://www.springeronline.com|}
\hbox{do a ``Springer Search" for the journal name}
\hbox{or a ``Springer Search -- By ISBN/ISSN"}}}

\medskip
If the journal's home page does not offer a particular \LaTeX\
package (i.e. a special class option for the SVJour document class)
begin your document with
\begin{verbatim}
   \documentclass[global]{svjour}
\end{verbatim}
This holds for a \emph{one-column} journal -- for a \emph{two-column}
journal just add the option \verb|twocolumn| so that the documentclass
command now reads
\begin{verbatim}
   \documentclass[global,twocolumn]{svjour}
\end{verbatim}
All other options are also described
in the main \emph{Users guide}.

\section{Things to Note}

There may be a slightly different layout for the journal you plan to
create an article for and which is not taken care for by the
\emph{global} option. This is not a great problem since all the articles
are reprocessed anyway to include final page numbers, bibliographic
information, and so on. The final layout is applied to your article
during this process thus deriving benefit from the general use of
\LaTeXe\ and the \SJour\ document class altogether.

Since Springer-Verlag publishes over 400 different journals -- over 50
thereof with \TeX\ -- it may take a while to produce a specific \TeX\
macro package just for the one you are writing for at the moment. Please
keep an eye on the web page mentioned above.
\end{document}
